\section{Related Work}
\label{sec:related}
Optimizing HW/SW partitioning for individual applications has received much research attention. For a limited design space, exact algorithms, such as integer linear programming \cite{kuang2005partitioning}, branch-and-bound \cite{jigang2004branch}, and dynamic programming \cite{wu2006low}, guarantee to find the optimal solution. For complex design spaces, this is an NP-hard problem, heuristics are the only viable approach (however without guaranteeing to find the optimum). Examples are genetic algorithms (GA) \cite{quan2014towards, alexandrescu2011genetic, wen2011heuristic, page2010multi}, simulated annealing \cite{liang2013hardware}, tabu search \cite{wu2013efficient}, and greedy algorithms \cite{tang2015hardware}. 

GA-based heuristics for design space exploration is a rich research area. Some approaches, e.g. \cite{quan2014towards}, keep the best solutions in each generation to orient the inheritance of good genes. GA with smart mutation \cite{alexandrescu2011genetic} uses platform knowledge to guide the mutation to increases the solution's convergence rate. The local search algorithm is also combined with GA \cite{wen2011heuristic} to accelerate the exploration speed. \cite{page2010multi} uses different heuristics initialization for getting better solutions. 
 
To accelerate the heuristic, a fast evaluation mechanism is essential. Existing heuristics either leverages analytical estimation \cite{pinedo2016scheduling, omara2010genetic}), or use abstract simulation and TLMs \cite{pimentel2006systematic, ghenassia2005transaction}. Some heuristics combine the benefits of analytical and simulation \cite{zhang2014automatic, mariani2010correlation}. They prune design space using analytical assumptions and then perform the simulation-based evaluation for the few design candidates.

Overall, existing DSE for creating new platforms focuses on a single application in isolation. Some ideas toward a domain focus can be extracted from platform-based computing which uses statistical information for destining the platform and application-specific mapping \cite{graf2014multi, gladigau2010system}. However more specialization for a wider set of applications is needed. Reconfigurable computing, such as \cite{wildermann2011operational}, aims to support multiple applications one at a time, however does need to rely of functional and structural similarities across reconfiguration cycles.
\newtext{Some promising architectures for domain-specific platforms have been proposed} (e.g. \cite{tabkhi2014function, nowatzki2017domain}). However, the domain analysis and domain DSE have not been tackled.

An early example of domain exploration is Domain Score Selection (DSS) \cite{zhang100ds} which proposes a greedy approach for identifying domain function kernels for hardware acceleration. While \cite{zhang100ds} is very promising in terms of formalization and definitions, the achieved performance however is bounded by limitations of the greedy approach.

%There is no clear path to broaden their scope to domain DSE.  which is not really accuracy only using simple score calculation to guide the approach. The domain DSE needs an efficient algorithms to explore the large domain design space. Fast and accuracy evaluation method to evaluate the efficiency of domain platform for all applications

