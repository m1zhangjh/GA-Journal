\begin{abstract}

Domain-specific computing is a promising solution to bridge the flexibility/efficiency gap for a broader set of applications. Streaming applications within a domain, such as video analytics,  software defined radio and radar, benefit from domain specialization due to functional and structural similarities. \newtext{However, current Design Space Exploration (DSE) focuses on individual applications in isolation. Hence, much of the domain optimization opportunities are missed. New DSE methodologies and tools are needed with a broader scope of application sets instead of individual applications.}

\newtext{This paper introduces a novel Domain-Specific DSE (DS-DSE) approach for domain-specific computing with a focus on streaming applications. Key contributions are: (1) a formalized method to extract the functional and structural similarities of domain applications, (2) rapid platform performance estimation and comparison in different abstract level, Domain Score (DS) and Analytical Model, (3) two novel algorithms, Dynamic Score Selection (DSS) and GenetIc Domain Exploration (GIDE), for hardware/software partitioning of a domain-specific platform to maximize the throughput across domain applications (under certain constraints) and (4) a methodology to evaluate a domain platform.}

\newtext{We demonstrate DSS's and \ga's benefits using OpenVX-based applications and synthetic domains. Our domain-specific platforms generated by DSS and \ga, achieve 58.02\%, and 74.85\% (OpenVX), 23.60\% and 48.09\% (synthetic) performance improvement compared to application-specific platform. The \ga's HW/SW mapping achieves 99.8\% (OpenVX) and 97.6\% (synthetic) throughput of the domain optimal platform (from exhaustive search).}

%Domain-specific computing is a promising solution to bridge the flexibility/efficiency gap for a broader set of applications. Streaming applications within a domain, such as video analytics,  software defined radio and radar, benefit from domain specialization due to functional and structural similarities. To aid their design, new Design Space Exploration (DSE) methodologies and tools are needed with a broader scope of application sets instead of individual applications. 
%Domain-specific DSE creates a systematic system design process to automate the design and architecture of domain-specific platforms.

%This paper introduces GenetIc Domain Exploration (GIDE), a novel Domain-Specific DSE (DS-DSE) to enable and accelerate the domain-specific platform design process. \ga is a Genetic Algorithm (GA) to determine the hardware/software partitioning for a group of applications within a domain to maximize the average throughput across all applications given a hardware budget. To cope with the enormous design space size, \ga accelerates evaluation with analytical estimation and traversal with a guided local search. 

%We demonstrate \ga's benefits using OpenVX-based applications and synthetic domains. The \ga's HW/SW mapping achieves 99.8\% (OpenVX) and 97.6\% (synthetic) throughput of the domain optimal platform (from exhaustive search). Compared to an earlier introduced greedy approach for DS-DSE, Domain Score Selection (DSS), \ga generated platforms achieve 10.65\% (OpenVX) and 19.81\% (synthetic) higher domain throughput. %add cite

\end{abstract}
