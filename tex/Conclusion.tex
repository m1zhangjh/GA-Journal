\section{Conclusion}

\label{sec:conclusion}
\newtext{This paper introduced a Domain-specific Design Space Exploration (DS-DSE) methodology to broaden the scope of existing DSE from individual apps in isolation to a whole domain. This paper lays the foundations by defining a domain and quantifiable features (metrics) that can guide exploration. The features take both behavioral (processing) and structural (communication) aspects into account, as well as consider the distribution over the domain.} 

\newtext{Based on these definitions, the paper has introduced the Dynamic Score Selection (DSS) and GenetIc Domain Exploration (GIDE) algorithms for domain exploration. These algorithms maximize domain throughput creates a domain-specific architecture that has more flexibility to execute domain applications. DSS uses Domain Score (DS) to estimate platform performance in the domain level.} \ga employs a guided local search with hybrid analytical and DS evaluation models to enhance the domain exploration efficiency. \newtext{Our experimental results (Synthetic and OpenVX domains) demonstrate that} DSS achieves 23.60\%-58.02\%, and \ga achieves 48.09\%-74.85\% performance improvement compared to application-specific FOP platform. \ga almost achieves domain optimal throughput (97.6\% to 99.8\%), with around $1*10^{13}$ faster compared with exhaustive search.

%This paper introduced a novel algorithm, GenetIc Domain Exploration (GIDE), for rapid Domain-specific Design Space Exploration (DS-DSE). The significance of \ga is broadening the scope of the genetic algorithm from a single application to a domain of applications. It employs a guided local search with hybrid analytical and domain score evaluation models to enhance the domain exploration efficiency. Our experimental results show that \ga almost achieves domain optimal throughput (97.6\% to 99.8\%), with around $1*10^{13}$ faster compared with exhaustive search. \ga also achieves 10.65\% to 19.81\% and 48.09\% to 74.85\% higher domain throughput  compared against domain-specific DSS \cite{zhang100ds} and application-specific FOP platforms.